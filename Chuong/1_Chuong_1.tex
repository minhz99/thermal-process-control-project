% !TeX root = ../main.tex

\documentclass[../main.tex]{subfiles}
\begin{document}

\section{Tổng quan về nhà máy nhiệt điện}

\subsection{Lịch sử phát triển}

Vào năm 1831, nhà khoa học Michael Faraday đã phát minh ra máy phát điện, đánh dấu bước ngoặt quan trọng trong lịch sử phát triển của loài người. Điện năng được tạo ra để phục vụ cho đời sống xã hội và ngày nay đã trở thành một phần không thể thiếu trong mọi lĩnh vực của cuộc sống. Gần như tất cả các thiết bị sản xuất hiện đại đều sử dụng điện năng làm nguồn năng lượng chính.

Tại Việt Nam, nhà máy nhiệt điện đầu tiên được người Pháp xây dựng tại Hà Nội với mục đích phục vụ cho chính quyền Pháp. Trong những năm 1980-1990 của thế kỷ XX, các nhà máy nhiệt điện của Việt Nam được xây dựng nhiều hơn với công suất ngày càng lớn. Gần đây, nhiều dự án nhiệt điện lớn đã được triển khai như Thái Bình, Vũng Áng, Cà Mau, Phú Mỹ. Tuy nhiên, do ảnh hưởng đến môi trường và sự cạn kiệt dần của nhiên liệu hóa thạch, cùng với xu hướng sử dụng năng lượng sạch của thế giới hiện nay, các dự án nhiệt điện mới hiện ít được triển khai hơn.

\subsection{Đặc điểm và cấu tạo}

Nhà máy nhiệt điện là nhà máy sử dụng các nguyên liệu hóa thạch như than đá, khí đốt hoặc nhiên liệu sinh học, năng lượng hạt nhân, năng lượng mặt trời để cung cấp nhiệt năng cho nước, tạo ra hơi nước làm quay tuabin và tạo ra dòng điện. Phần hơi nước sau khi đi qua tuabin sẽ được ngưng tụ và thu hồi để tái sử dụng cho các chu trình tiếp theo, tạo nên một chu trình khép kín.

Cấu tạo sơ bộ của nhà máy nhiệt điện gồm các bộ phận chính. Lò hơi có nhiệm vụ chuyển đổi nước thành hơi nước thông qua quá trình đốt cháy nhiên liệu, đồng thời trong giai đoạn này năng lượng hóa học cũng được chuyển đổi thành nhiệt năng. Bình ngưng dùng để ngưng tụ lượng hơi thoát từ tuabin hạ áp thành nước ngưng cung cấp cho chu trình nhiệt, ngoài ra còn đảm nhận vai trò cung cấp nước cho hệ thống và góp phần đảm bảo quá trình sản xuất năng lượng điện diễn ra an toàn và hiệu quả. Tuabin thu thập hơi nước để làm quay các cánh quạt nhằm tạo ra dòng điện xoay chiều, thường có nhiều thân với áp suất đa dạng để tận dụng tối đa hơi nước. Bơm cấp cung cấp nước trong hệ thống nhà máy nhiệt điện, bơm nước đến vị trí cần thiết, đồng thời giữ được sự ổn định của dòng chảy, tạo sự ổn định cho toàn nhà máy khi vận hành.

% \begin{figure}[H]
% \centering
% % \includegraphics[width=0.9\textwidth]{Hinhve/chuong1_mo_hinh_nha_may.png}
% \caption{Mô hình nhà máy nhiệt điện}
% \label{fig:ch1_mo_hinh_nha_may}
% \end{figure}

\begin{figure}[H]
\centering
\includegraphics[width=1\textwidth]{Hinhve/so-do-nha-may.png}
\caption{Sơ đồ cấu tạo sơ bộ của nhà máy nhiệt điện}
\label{fig:ch1_so_do_cau_tao}
\end{figure}

\subsection{Nguyên lý hoạt động}

Về cơ bản, nhà máy nhiệt điện hoạt động dựa trên nguyên lý biến đổi nhiệt năng sang cơ năng thông qua việc đốt cháy nhiên liệu hóa thạch, sau đó chuyển hóa từ cơ năng thành điện năng thông qua việc hơi nước làm quay tuabin gắn với máy phát điện. Nguyên liệu đầu vào là than đá được nghiền thành bột mịn bằng máy nghiền và được thổi vào lò hơi để đốt cháy. Bên trong lò hơi, phần nhiệt được tạo ra sẽ chuyển hóa nước thành hơi nước.

Dưới điều kiện áp suất cao, hơi nước sẽ làm quay cánh tuabin được nối với máy phát điện, kết quả là máy phát điện tạo ra điện năng. Phần hơi nước sau đó sẽ được làm nguội, ngưng tụ và tuần hoàn trở lại lò hơi để chuyển hóa thành hơi nước, cung cấp năng lượng cho tuabin trong chu trình liên tục.

\section{Hơi quá nhiệt và vai trò trong nhà máy nhiệt điện}

\subsection{Định nghĩa và nguyên lý hình thành}

Hơi quá nhiệt là trạng thái hơi có nhiệt độ cao hơn nhiệt độ hơi bão hòa tại cùng áp suất. Khi gia nhiệt nước lỏng đến nhiệt độ nhất định, nước sẽ chuyển sang trạng thái hơi bão hòa. Lúc này, hơi bão hòa vẫn có thể chứa pha lỏng, tức là nước dạng lỏng và được gọi là hơi bão hòa ẩm. Các giá trị nhiệt độ phổ biến của hơi quá nhiệt thường nằm trong khoảng 380-540°C.

Nguyên lý hình thành hơi quá nhiệt được thực hiện bằng cách gia nhiệt cho hơi ướt hoặc hơi bão hòa để vượt ra khỏi điểm bão hòa. Ở trạng thái này, hơi nước có nhiệt độ cao hơn và mật độ thấp hơn so với điểm bão hòa trong cùng môi trường áp suất. Từ hơi bão hòa, người ta sử dụng thêm bộ gia nhiệt để thu được hơi quá nhiệt phục vụ cho các ứng dụng công nghiệp. Khi lò hoạt động, nhiên liệu được đốt cháy trong buồng đốt, làm cho nước nóng lên và sôi ở nhiệt độ nhất định. Khi nước sôi và không thể tăng nhiệt độ thêm nữa thì sẽ chuyển hóa thành hơi bão hòa. Hơi bão hòa được đưa qua bộ gia nhiệt bên ngoài và tạo thành hơi quá nhiệt.

\subsection{Ứng dụng của hơi quá nhiệt}

Hơi quá nhiệt đã được làm khô một cách hiệu quả, tăng nhiệt độ đến một điểm mà khả năng ngưng tụ giảm đáng kể và làm tăng thể tích một cách đáng kể. Những yếu tố này làm tăng sức mạnh và tính kinh tế của hệ thống động lực. Tuy nhiên, nhược điểm chính của hơi quá nhiệt nằm ở độ phức tạp và chi phí tăng thêm của hệ thống ống dẫn siêu nhiệt, cũng như ảnh hưởng bất lợi mà hơi khô gây ra đối với việc bôi trơn các bộ phận chuyển động như van hơi.

Các ứng dụng tiềm năng khác của hơi quá nhiệt bao gồm làm khô, làm sạch, phân lớp, kỹ thuật phản ứng, làm khô epoxy, công nghệ cứng, hệ thống năng lượng và công nghệ nano. Hơi nước siêu nhiệt cũng được ứng dụng trong việc khử trùng môi trường nhà máy chế biến thực phẩm khô. Trong công nghiệp tinh chế và hydrocacbon, hơi nước siêu nhiệt được sử dụng chủ yếu cho các mục đích tẩy rửa và làm sạch. Tuy nhiên, hơi quá nhiệt thường không được sử dụng làm chất trao đổi nhiệt do có hệ số dẫn nhiệt thấp.

Mặc dù có nhiều ứng dụng, hơi quá nhiệt dễ làm hư hỏng và hao mòn thiết bị do nhiệt độ rất cao và hoàn toàn khô. Ngoài ra, do phải sử dụng bộ phận gia nhiệt để tăng nhiệt độ của hơi nên chi phí sản xuất tăng cao.

\begin{figure}[H]
\centering
\includegraphics[width=1\textwidth]{Hinhve/hinh_anh_tuabin.jpg}
\caption{Hình ảnh rôto tuabin trong nhà máy nhiệt điện}
\label{fig:ch1_canh_tuabin}
\end{figure}

\subsection{Vai trò trong nhà máy nhiệt điện}

Trong chu trình hoạt động của nhà máy nhiệt điện, nước được gia nhiệt một phần từ bộ hâm nước rồi được đưa lên lò hơi để tiếp tục gia nhiệt. Sau khi lò hơi gia nhiệt xong, nước sẽ được đưa lên bao hơi, nơi thực hiện nhiệm vụ phân li hơi ra khỏi hỗn hợp nước. Hơi lúc này là hơi bão hòa, và hơi này sẽ được đưa qua bộ quá nhiệt, khi đó hơi bão hòa sẽ trở thành hơi quá nhiệt. Hơi quá nhiệt sau đó được đưa vào tuabin để làm cho các cánh quạt tuabin quay, tạo ra dòng điện xoay chiều.

Từ đó có thể thấy rằng vai trò của hơi quá nhiệt trong nhà máy nhiệt điện là vô cùng quan trọng. Nó mang lại nguồn năng lượng lớn cho tuabin đạt tốc độ yêu cầu và giúp bảo vệ tuabin khỏi hiện tượng thủy kích do ngưng tụ hơi ẩm. Việc duy trì nhiệt độ hơi quá nhiệt ổn định không chỉ đảm bảo hiệu suất chuyển đổi năng lượng cao mà còn kéo dài tuổi thọ thiết bị. Hiện nay trong các nhà máy nhiệt điện, quá trình điều khiển tự động nhiệt độ hơi quá nhiệt đã rất phổ biến và đóng vai trò then chốt trong hoạt động của toàn bộ hệ thống.

\begin{figure}[H]
\centering
\includegraphics[width=1\textwidth]{Hinhve/chuong1_hoi_qua_nhiet.png}
\caption{Hơi trong lò hơi}
\label{fig:ch1_hoi_qua_nhiet}
\end{figure}

\section{Các phương pháp điều khiển hơi quá nhiệt}

Hiện nay có hai phương pháp điều khiển hơi quá nhiệt phổ biến thường được sử dụng là điều chỉnh nhiệt độ hơi quá nhiệt bằng hơi và điều chỉnh nhiệt độ hơi quá nhiệt bằng khói. Ở phương pháp điều chỉnh bằng hơi, nhiệt độ hơi quá nhiệt được điều chỉnh nhờ tác dụng của nguồn nước hay nguồn hơi. Ở phương pháp điều chỉnh bằng khói, nhiệt độ hơi quá nhiệt được điều chỉnh bằng cách thay đổi lượng nhiệt hấp thụ của bộ quá nhiệt do thay đổi lưu lượng khói hay thay đổi nhiệt độ khói.

\subsection{Điều chỉnh bằng hơi}

Ở phương pháp này, người ta đặt vào ống lò hơi của bộ quá nhiệt một thiết bị gọi là bộ giảm ôn. Nước đi qua bộ giảm ôn sẽ nhận nhiệt của hơi làm cho nhiệt độ hơi quá nhiệt giảm xuống do nước có nhiệt độ thấp hơn hơi. Khi thay đổi lưu lượng nước qua bộ giảm ôn thì sẽ làm thay đổi nhiệt độ hơi quá nhiệt. Hiện nay thường được sử dụng hai loại bộ giảm ôn là bộ giảm ôn kiểu bề mặt và bộ giảm ôn kiểu hỗn hợp.

\begin{figure}[H]
    \centering
    \includegraphics[width=0.6\textwidth]{Hinhve/chuong1_bo_tri_giam_on.png}
    \caption{Các cách bố trí bộ giảm ôn bề mặt}
    \label{fig:ch1_bo_tri_giam_on}
    \end{figure}

Bộ giảm ôn kiểu bề mặt có thể đặt ở các vị trí khác nhau trong bộ quá nhiệt như đầu ra của bộ quá nhiệt, đầu vào của bộ quá nhiệt, hoặc xen kẽ giữa hai cấp bộ quá nhiệt. Nếu bố trí đặt ở đầu vào thì sẽ điều được nhiệt độ trong toàn bộ quá nhiệt, nhưng có nhược điểm là quán tính nhiệt lớn, tác động chậm. Nếu bố trí đặt ở đầu ra bộ quá nhiệt thì quán tính điều chỉnh nhiệt bé, tuabin được bảo vệ tuyệt đối, nhưng bộ quá nhiệt không được bảo vệ. Để khắc phục nhược điểm trên, thường người ta bố trí bộ giảm ôn nằm giữa hai cấp của bộ quá nhiệt.

\begin{figure}[H]
    \centering
    \includegraphics[width=0.9\textwidth]{Hinhve/chuong1_cau_tao_giam_on.png}
    \caption{Cấu tạo bộ giảm ôn kiểu bề mặt}
    \label{fig:ch1_cau_tao_giam_on}
    \end{figure}

Bộ giảm ôn bề mặt được sử dụng rộng rãi hiện nay ở những lò hơi áp suất trung bình và cao do không đòi hỏi yêu cầu cao về chất lượng nguồn nước để giảm ôn. Về cấu tạo, nó là một thiết bị trao đổi nhiệt kiểu ống, trong đó nước để giảm ôn đi trong những ống đồng uốn hình chữ U còn hơi đi ngoài ống. Nước dùng để giảm ôn có thể là nước cấp hay nước lò. Bộ giảm ôn bề mặt được thiết kế để bảo đảm khoảng điều chỉnh nhiệt độ hơi là 40-50°C tương ứng với lượng nước qua giảm ôn vào khoảng 40-60\% lưu lượng nước cấp.

\begin{figure}[H]
\centering
\includegraphics[width=0.85\textwidth]{Hinhve/chuong1_so_do_noi_giam_on.png}
\caption{Sơ đồ nối bộ giảm ôn với đường nước cấp}
\label{fig:ch1_so_do_noi_giam_on}
\end{figure}

Bộ giảm ôn kiểu hỗn hợp làm việc trên cơ sở hơi quá nhiệt được làm lạnh bằng dòng nước phun vào qua các ống phun. Các hạt nước được phun thành bụi nhỏ rồi hỗn hợp với hơi quá nhiệt và bốc hơi, đồng thời làm giảm nhiệt độ hơi quá nhiệt. Khác với giảm ôn bề mặt, ở giảm ôn hỗn hợp nước dùng để giảm ôn đòi hỏi rất cao về chất lượng, thường dùng là nước ngưng có chất lượng cao hay nước đã xử lý tốt.

Ưu điểm chủ yếu của giảm ôn hỗn hợp là quán tính điều chỉnh tương đối bé (30-60 giây), cấu tạo đơn giản và làm việc chắc chắn. Nước có thể phun trực tiếp trong ống dẫn hơi hay riêng trong bộ giảm ôn hỗn hợp. Khi phun trực tiếp nước trong các ống dẫn hơi cần bảo đảm khoảng cách thẳng từ 5-8 m kể từ mũi phun nước để xáo trộn tốt nước phun với hơi quá nhiệt và để bốc hơi hoàn toàn nước phun này.

Giảm ôn hỗn hợp được sử dụng rộng rãi trong các lò hơi lớn hiện đại, nhưng chỉ dùng cho bộ quá nhiệt sơ cấp. Đối với bộ quá nhiệt trung gian thì không được sử dụng vì khi phun nước vào bộ quá nhiệt trung gian, sẽ làm tăng lưu lượng hơi quá nhiệt ra khỏi bộ quá nhiệt trung gian và do đó làm tăng lưu lượng hơi qua phần trung áp và hạ áp tuabin, dẫn tới làm giảm lưu lượng hơi phần cao áp, đây là điều không kinh tế.

\begin{figure}[H]
\centering
\includegraphics[width=0.85\textwidth]{Hinhve/chuong1_giam_on_hon_hop.png}
\caption{Sơ đồ cấu tạo bộ giảm ôn hỗn hợp}
\label{fig:ch1_giam_on_hon_hop}
\end{figure}

\subsection{Điều chỉnh bằng khói}

Điều chỉnh nhiệt độ hơi quá nhiệt bằng khói được thực hiện bằng cách thay đổi lượng nhiệt hấp thụ của bộ quá nhiệt thông qua ba biện pháp chính là thay đổi lưu lượng khói đi qua bộ quá nhiệt, thay đổi nhiệt độ khói đi qua bộ quá nhiệt, và thay đổi đồng thời lưu lượng cùng nhiệt độ khói.

Biện pháp thay đổi lưu lượng khói được thực hiện bằng cách cho một phần khói đi tắt qua đường khói không đặt bộ quá nhiệt nhằm giảm lượng nhiệt mà bộ quá nhiệt nhận được, do đó làm giảm nhiệt độ quá nhiệt. Có nhiều dạng đường khói đi tắt khác nhau như có toàn bộ đường khói đi tắt, có một phần đường khói đi tắt, hoặc có đặt bề mặt hấp thụ nhiệt trong đường khói đi tắt.

\begin{figure}[H]
\centering
\includegraphics[width=1\textwidth]{Hinhve/chuong1_duong_khoi_di_tat.png}
\caption{Các dạng đường khói đi tắt của bộ quá nhiệt}
\label{fig:ch1_duong_khoi_di_tat}
\end{figure}

Biện pháp thay đổi nhiệt độ khói được thực hiện bằng cách thay đổi góc quay của vòi phun, cho vòi phun hướng lên trên hoặc xuống dưới làm thay đổi vị trí trung tâm của ngọn lửa, do đó làm thay đổi nhiệt độ khói ra khỏi buồng lửa, tức là thay đổi nhiệt độ khói đi qua bộ quá nhiệt, dẫn đến thay đổi nhiệt độ hơi quá nhiệt.

\begin{figure}[H]
\centering
\includegraphics[width=0.75\textwidth]{Hinhve/chuong1_vi_tri_ngon_lua.png}
\caption{Điều chỉnh nhiệt độ hơi quá nhiệt bằng cách thay đổi vị trí ngọn lửa}
\label{fig:ch1_vi_tri_ngon_lua}
\end{figure}

Biện pháp thay đổi đồng thời lưu lượng và nhiệt độ khói được thực hiện bằng cách trích một phần khói ở phía sau bộ hâm nước đưa vào buồng lửa (còn gọi là tái tuần hoàn khói). Khi trích một phần khói ở phía sau bộ hâm nước đưa vào buồng lửa, nhiệt độ trung bình trong buồng lửa sẽ giảm xuống làm cho nhiệt lượng hấp thụ bằng bức xạ của dàn ống sinh hơi giảm xuống, nghĩa là nhiệt độ khói ra khỏi buồng lửa tăng lên, trong khi đó lưu lượng khói đi qua bộ quá nhiệt tăng lên làm cho lượng nhiệt hấp thụ của bộ quá nhiệt tăng lên, dẫn đến nhiệt độ hơi quá nhiệt cũng tăng lên. Nhược điểm chủ yếu của phương pháp tái tuần hoàn khói là phải đặt thêm thiết bị quạt, do đó tốn thêm điện năng tự dùng.

\begin{figure}[H]
\centering
\includegraphics[width=0.8\textwidth]{Hinhve/chuong1_tai_tuan_hoan_khoi.png}
\caption{Điều chỉnh nhiệt độ hơi quá nhiệt bằng cách tái tuần hoàn khói}
\label{fig:ch1_tai_tuan_hoan_khoi}
\end{figure}

Nhìn chung, các phương pháp điều chỉnh bằng khói thường có quán tính điều chỉnh không rộng, đôi khi hiệu quả kinh tế lại giảm đi. Việc tự động hóa các hệ thống điều chỉnh bằng khói thường khó khăn hơn bằng hơi nên phương pháp này ít được sử dụng hơn phương pháp điều chỉnh bằng hơi. Qua việc phân tích các phương pháp điều chỉnh nhiệt độ hơi quá nhiệt, có thể thấy mỗi phương pháp có những ưu nhược điểm nhất định. Vì vậy trong lò hơi hiện đại, người ta hay dùng phối hợp nhiều biện pháp như dùng phun hơi kết hợp với vòi phun quay hay với tái tuần hoàn khói, hoặc sử dụng đồng thời giảm ôn bề mặt đặt ở phía hơi bão hòa và giảm ôn hỗn hợp đặt xen kẽ bộ quá nhiệt.

\subsection{Sơ đồ hệ thống điều khiển}

Hệ thống điều khiển nhiệt độ hơi quá nhiệt trong thực tế thường áp dụng cấu trúc điều khiển cascade hai vòng để nâng cao chất lượng điều chỉnh. Sơ đồ mạch vòng điều khiển chính của hệ thống bao gồm cảm biến nhiệt độ đo nhiệt độ ở sau bộ quá nhiệt và gửi tín hiệu đến bộ điều khiển, bộ phun nước giảm ôn xé nhỏ các hạt nước sau đó phun vào dòng hơi để điều chỉnh nhiệt độ hơi quá nhiệt, và bộ điều khiển PID (Proportional Integral Derivative) là cơ chế phản hồi vòng điều khiển được sử dụng rộng rãi trong các hệ thống điều khiển công nghiệp.

\begin{figure}[H]
\centering
\includegraphics[width=0.85\textwidth]{Hinhve/chuong1_so_do_mach_vong.png}
\caption{Sơ đồ mạch vòng điều khiển chính của hệ thống}
\label{fig:ch1_so_do_mach_vong}
\end{figure}

Bộ điều khiển PID tính toán giá trị sai số là hiệu số giữa giá trị đo thông số biến đổi và giá trị đặt mong muốn, sau đó thực hiện giảm tối đa sai số bằng cách điều chỉnh giá trị điều khiển đầu vào. Nguyên lý hoạt động của hệ thống điều khiển như sau: tín hiệu nhiệt độ hơi quá nhiệt được lấy ở sau bộ quá nhiệt rồi truyền đến bộ PID, tại đây nó sẽ được so sánh với giá trị đặt. Nếu có sai lệch giữa giá trị đặt và tín hiệu nhiệt độ thì bộ điều khiển PID sẽ đưa ra tín hiệu điều chỉnh đến đối tượng điều khiển là van phun nước giảm ôn để điều chỉnh độ đóng mở van, từ đó điều chỉnh nhiệt độ hơi quá nhiệt về giá trị mong muốn.

\begin{figure}[H]
    \centering
    \includegraphics[width=0.9\textwidth]{Hinhve/chuong1_cau_truc_dieu_khien.png}
    \caption{Sơ đồ cấu trúc điều khiển hệ thống}
    \label{fig:ch1_cau_truc_dieu_khien}
    \end{figure}

Ưu điểm của thiết kế hai vòng điều khiển là đảm bảo an toàn trong quá trình vận hành và độ nhanh nhạy đã được cải thiện. Với sự ảnh hưởng của nhiễu được mạch vòng trong kiểm soát nhanh chóng để gửi tín hiệu điều chỉnh kịp thời. Tuy nhiên, nhược điểm là cấu trúc phức tạp hơn do sử dụng thiết kế mạch hai vòng với hai bộ PID. Trong thực tế triển khai, PLC S7-1200 kết hợp với TIA Portal cung cấp nền tảng lý tưởng để triển khai các thuật toán điều khiển này.

\section{Tổng quan về PLC S7-1200}

\subsection{Giới thiệu chung}

Bộ điều khiển logic khả trình PLC (Programmable Logic Controller) là thiết bị điều khiển công nghiệp được thiết kế để thực hiện các chức năng điều khiển tự động trong môi trường công nghiệp khắc nghiệt. Ra đời từ những năm 1960 với mục đích ban đầu là thay thế các hệ thống rơ-le điện cứng, PLC đã trải qua quá trình phát triển mạnh mẽ và trở thành thiết bị không thể thiếu trong tự động hóa công nghiệp hiện đại. Qua nhiều thập kỷ cải tiến, PLC hiện đại không chỉ thực hiện các chức năng logic đơn giản mà còn có khả năng xử lý các thuật toán điều khiển phức tạp, truyền thông mạng công nghiệp và tích hợp với các hệ thống giám sát cấp cao.

\begin{figure}[H]
\centering
\includegraphics[width=0.6\textwidth]{Hinhve/plc_siemens_s7_1200.jpg}
\caption{PLC Siemens S7-1200}
\label{fig:plc_siemens_s7_1200}
\end{figure}

Trong số các nhà sản xuất PLC hàng đầu thế giới, Siemens AG của Đức là một trong những đơn vị tiên phong với dòng sản phẩm SIMATIC đa dạng và chất lượng cao. Dòng PLC SIMATIC S7 của Siemens bao gồm nhiều dòng sản phẩm khác nhau nhằm đáp ứng nhu cầu từ các ứng dụng nhỏ đến các hệ thống tự động hóa quy mô lớn và phức tạp. Trong đó, dòng S7-1200 được định vị như một giải pháp điều khiển cho các ứng dụng từ quy mô nhỏ đến trung bình, kết hợp giữa tính nhỏ gọn, hiệu suất cao và khả năng mở rộng linh hoạt.

\subsection{Đặc điểm kỹ thuật}

PLC S7-1200 được phát triển dựa trên kiến trúc modular linh hoạt, cho phép người dùng cấu hình hệ thống phù hợp với yêu cầu cụ thể của từng ứng dụng. CPU của S7-1200 có nhiều phiên bản khác nhau với hiệu năng xử lý từ cơ bản đến nâng cao, đáp ứng các yêu cầu điều khiển từ đơn giản đến phức tạp. Các CPU thường được trang bị bộ nhớ chương trình và bộ nhớ dữ liệu có dung lượng từ vài chục KB đến vài trăm KB, đủ để lưu trữ các chương trình điều khiển phức tạp và dữ liệu vận hành.

Về khả năng xử lý, CPU S7-1200 có tốc độ xử lý nhanh với thời gian thực thi lệnh tính bằng micro giây, đảm bảo đáp ứng kịp thời các yêu cầu điều khiển trong thời gian thực. Bộ nhớ làm việc của PLC được chia thành nhiều vùng khác nhau bao gồm vùng nhớ bit, vùng nhớ byte, vùng nhớ từ, vùng nhớ kép từ và vùng nhớ dữ liệu, cho phép tổ chức và quản lý dữ liệu một cách hiệu quả. Đặc biệt, S7-1200 hỗ trợ chức năng giữ dữ liệu (retentive memory) giúp bảo toàn thông tin quan trọng ngay cả khi mất nguồn điện.

Hệ thống đầu vào đầu ra của S7-1200 được thiết kế linh hoạt với CPU tích hợp sẵn một số lượng nhất định các điểm I/O và có thể mở rộng thông qua các module bổ sung. Các đầu vào số có thể nhận tín hiệu từ các thiết bị như công tắc, cảm biến tiệm cận, nút nhấn với điện áp làm việc tiêu chuẩn 24VDC. Các đầu ra số có khả năng điều khiển các thiết bị chấp hành như van điện từ, động cơ, đèn báo với dòng điện tải phù hợp. Đối với các ứng dụng cần xử lý tín hiệu analog, S7-1200 hỗ trợ các module đầu vào analog với độ phân giải cao, thường là 12 bit hoặc 16 bit, cho phép đo chính xác các đại lượng liên tục như nhiệt độ, áp suất, lưu lượng.

\subsection{Khả năng truyền thông}

Một trong những ưu điểm nổi bật của PLC S7-1200 là khả năng truyền thông mạng mạnh mẽ và đa dạng. CPU S7-1200 được tích hợp sẵn cổng Ethernet PROFINET, hỗ trợ giao thức truyền thông công nghiệp tiêu chuẩn, cho phép kết nối dễ dàng với các thiết bị khác trong hệ thống tự động hóa. Thông qua PROFINET, PLC có thể truyền thông với các thiết bị HMI (Human Machine Interface) để hiển thị và điều khiển, kết nối với các PLC khác để tạo thành hệ thống phân tán, và tích hợp với các hệ thống SCADA (Supervisory Control and Data Acquisition) cấp cao hơn.

Ngoài PROFINET, S7-1200 còn hỗ trợ nhiều giao thức truyền thông khác như Modbus TCP/IP, cho phép tương thích với các thiết bị từ nhiều nhà sản xuất khác nhau. Khả năng truyền thông mở rộng này giúp S7-1200 dễ dàng tích hợp vào các hệ thống hiện có và tương lai, đảm bảo tính linh hoạt và khả năng nâng cấp của hệ thống. Tốc độ truyền thông Ethernet lên đến 100 Mbps đảm bảo trao đổi dữ liệu nhanh chóng và đáng tin cậy giữa các thiết bị trong mạng.

\subsection{Môi trường lập trình TIA Portal}

TIA Portal (Totally Integrated Automation Portal) là môi trường phát triển tích hợp do Siemens cung cấp cho việc lập trình, cấu hình và vận hành các thiết bị tự động hóa. Đây là một nền tảng phần mềm thống nhất, cho phép người dùng thực hiện tất cả các công việc liên quan đến tự động hóa từ một giao diện duy nhất, bao gồm lập trình PLC, thiết kế giao diện HMI, cấu hình mạng truyền thông và chẩn đoán hệ thống.

\begin{figure}[H]
\centering
\includegraphics[width=0.78\textwidth]{Hinhve/moi-truong-lap-trinh-tia-portal.jpg}
\caption{Môi trường lập trình TIA Portal}
\label{fig:moi_truong_tia_portal}
\end{figure}


Đối với PLC S7-1200, TIA Portal cung cấp nhiều ngôn ngữ lập trình tuân theo tiêu chuẩn IEC 61131-3, bao gồm Ladder Diagram (LAD) dạng sơ đồ thang phù hợp với kỹ sư điện, Function Block Diagram (FBD) dạng sơ đồ khối chức năng thuận tiện cho lập trình các thuật toán phức tạp, Statement List (STL) dạng danh sách lệnh cho lập trình cấp thấp, và Structured Control Language (SCL) ngôn ngữ cấu trúc tương tự ngôn ngữ lập trình bậc cao. Sự đa dạng này cho phép lập trình viên lựa chọn ngôn ngữ phù hợp nhất với từng phần của ứng dụng và tận dụng kinh nghiệm sẵn có.

TIA Portal không chỉ là công cụ lập trình mà còn tích hợp các chức năng hỗ trợ mạnh mẽ khác. Chức năng mô phỏng (simulation) cho phép kiểm tra và gỡ lỗi chương trình trước khi triển khai thực tế, tiết kiệm thời gian và chi phí. Các công cụ chẩn đoán tích hợp giúp phát hiện và khắc phục nhanh chóng các lỗi phần cứng và phần mềm trong quá trình vận hành. Hệ thống quản lý phiên bản giúp theo dõi và kiểm soát các thay đổi trong dự án, đặc biệt hữu ích trong các dự án lớn và phức tạp.

\subsection{Ứng dụng trong điều khiển nhiệt điện}

Trong ứng dụng điều khiển nhiệt độ hơi quá nhiệt tại nhà máy nhiệt điện, PLC S7-1200 đóng vai trò là bộ não trung tâm của hệ thống điều khiển tự động. PLC nhận tín hiệu từ các cảm biến nhiệt độ, áp suất và lưu lượng, xử lý các tín hiệu này thông qua các thuật toán điều khiển được lập trình sẵn, và đưa ra tín hiệu điều khiển đến các van điều khiển và thiết bị chấp hành khác. Khả năng xử lý analog chính xác của S7-1200 đặc biệt quan trọng trong việc đọc các tín hiệu cảm biến nhiệt độ có độ phân giải cao và điều khiển chính xác vị trí van.

Các khối chức năng PID tích hợp sẵn trong S7-1200 cho phép triển khai dễ dàng các thuật toán điều khiển PID cổ điển, là nền tảng cho hầu hết các ứng dụng điều khiển quá trình. Ngoài ra, người dùng có thể lập trình các thuật toán điều khiển nâng cao hơn như điều khiển cascade, điều khiển tiến, hoặc các phương pháp điều khiển hiện đại khác phù hợp với đặc tính của đối tượng điều khiển nhiệt. Khả năng lưu trữ và xử lý dữ liệu lịch sử của PLC cũng hỗ trợ việc phân tích xu hướng và tối ưu hóa vận hành hệ thống.

Tính tin cậy cao của S7-1200 trong môi trường công nghiệp khắc nghiệt là yếu tố quan trọng đối với các ứng dụng trong nhà máy nhiệt điện. PLC được thiết kế để hoạt động ổn định trong điều kiện nhiệt độ môi trường rộng, chịu được nhiễu điện từ và rung động cơ học. Các cơ chế tự chẩn đoán và xử lý lỗi tích hợp giúp phát hiện sớm các vấn đề tiềm ẩn và đảm bảo tính liên tục của quá trình sản xuất. Khả năng truyền thông với hệ thống SCADA cấp cao hơn cho phép giám sát và điều khiển từ xa, tạo điều kiện cho vận hành tập trung và hiệu quả.

\section{Phân tích thông số công nghệ}

\subsection{Quá trình sản xuất hơi quá nhiệt}

Quá trình sản xuất hơi quá nhiệt trong nhà máy nhiệt điện liên quan đến nhiều giai đoạn chuyển đổi năng lượng. Hơi bão hòa từ lò hơi được đưa vào bộ quá nhiệt, nơi nhiệt độ của hơi được nâng lên cao hơn nhiệt độ bão hòa tương ứng với áp suất làm việc. Quá trình này loại bỏ hoàn toàn các giọt nước lơ lửng trong hơi, đồng thời tăng entanpi của hơi, từ đó nâng cao hiệu suất nhiệt của chu trình nhiệt động. Đối với các nhà máy nhiệt điện công suất lớn, nhiệt độ hơi quá nhiệt thường được duy trì trong khoảng từ 500 đến 600°C, tùy thuộc vào công nghệ và thiết kế cụ thể của từng nhà máy.

Các thông số công nghệ quan trọng cần được giám sát và kiểm soát bao gồm nhiệt độ hơi quá nhiệt tại đầu ra của bộ quá nhiệt, áp suất hơi trước và sau bộ quá nhiệt, lưu lượng hơi đi qua, cũng như nhiệt độ khói lò. Sự biến động của bất kỳ thông số nào cũng có thể ảnh hưởng đến chất lượng hơi quá nhiệt và hiệu suất vận hành của tuabin. Do đó, việc thiết lập hệ thống đo lường và giám sát chính xác các thông số này là điều kiện tiên quyết để xây dựng hệ thống điều khiển hiệu quả.

% \begin{figure}[H]
% \centering
% \includegraphics[width=0.8\textwidth]{Hinhve/sodoLNG.png}
% \caption{Sơ đồ khối quá trình sản xuất hơi quá nhiệt}
% \label{fig:ch1_qua_trinh_san_xuat_hoi}
% \end{figure}

\subsection{Các thông số kỹ thuật chính}

Nhiệt độ hơi quá nhiệt là thông số trung tâm cần được điều khiển chính xác. Giá trị này phụ thuộc vào nhiều yếu tố như lưu lượng nhiên liệu cung cấp cho lò đốt, lưu lượng hơi đi qua bộ quá nhiệt, nhiệt độ khói lò và vị trí phun nước giảm nhiệt. Trong điều kiện vận hành bình thường, nhiệt độ hơi quá nhiệt cần được duy trì ổn định với độ dao động không vượt quá 5-10°C so với giá trị đặt. Sự dao động quá mức có thể dẫn đến ứng suất nhiệt trong các bộ phận của tuabin, gây ra hiện tượng mỏi nhiệt và giảm tuổi thọ thiết bị.

Áp suất hơi là thông số quan trọng khác ảnh hưởng trực tiếp đến quá trình quá nhiệt. Sự thay đổi áp suất không chỉ làm thay đổi nhiệt độ bão hòa mà còn ảnh hưởng đến hệ số truyền nhiệt trong bộ quá nhiệt. Lưu lượng hơi, bên cạnh việc quyết định công suất phát điện, còn là yếu tố quan trọng trong việc xác định lượng nhiệt cần thiết để đạt được nhiệt độ quá nhiệt mong muốn. Khi tải của nhà máy thay đổi, lưu lượng hơi cũng biến đổi tương ứng, đòi hỏi hệ thống điều khiển phải có khả năng thích ứng nhanh để duy trì nhiệt độ ổn định.

% \begin{figure}[H]
% \centering
% \includegraphics[width=0.75\textwidth]{Hinhve/chuong1_thong_so_ky_thuat.png}
% \caption{Các thông số kỹ thuật quan trọng của hệ thống hơi quá nhiệt}
% \label{fig:ch1_thong_so_ky_thuat}
% \end{figure}

\section{Đặc tính động học của đối tượng}

\subsection{Cơ chế truyền nhiệt và động học}

Quá trình truyền nhiệt trong bộ quá nhiệt diễn ra phức tạp với nhiều cơ chế xảy ra đồng thời. Nhiệt từ khói lò được truyền sang hơi thông qua bức xạ và đối lưu, trong khi hơi chảy qua các ống quá nhiệt trao đổi nhiệt với thành ống theo cơ chế đối lưu cưỡng bức. Đặc tính động học của quá trình này thể hiện qua độ trễ thời gian đáng kể giữa tác động điều khiển và phản hồi của nhiệt độ hơi. Thời gian trễ này có thể lên đến vài chục giây hoặc thậm chí vài phút, tùy thuộc vào kích thước và cấu tạo của bộ quá nhiệt.

Hằng số thời gian của đối tượng điều khiển nhiệt độ hơi quá nhiệt thường có giá trị lớn, phản ánh quán tính nhiệt của toàn bộ hệ thống. Khi có tác động điều khiển, nhiệt độ hơi không thay đổi ngay lập tức mà cần một khoảng thời gian nhất định để đạt đến giá trị mới. Hơn nữa, đối tượng này còn thể hiện tính chất phi tuyến rõ rệt, đặc biệt khi vận hành ở các chế độ tải khác nhau. Hệ số khuếch đại của đối tượng có thể thay đổi đáng kể khi điểm làm việc dịch chuyển, đòi hỏi hệ thống điều khiển phải có khả năng thích ứng với những thay đổi này.

% \begin{figure}[H]
% \centering
% % \includegraphics[width=0.8\textwidth]{Hinhve/chuong1_dac_tinh_dong_hoc.png}
% \caption{Đặc tính động học của quá trình quá nhiệt hơi}
% \label{fig:ch1_dac_tinh_dong_hoc}
% \end{figure}

\subsection{Ảnh hưởng của các yếu tố nhiễu}

Hệ thống điều khiển nhiệt độ hơi quá nhiệt phải đối mặt với nhiều nguồn nhiễu có thể gây dao động nhiệt độ. Trong số đó, sự thay đổi tải của tuabin là nguồn nhiễu chính và thường xuyên nhất. Khi tải giảm, lưu lượng hơi giảm theo, dẫn đến nhiệt độ hơi tăng cao nếu không có biện pháp điều chỉnh kịp thời. Ngược lại, khi tải tăng đột ngột, nhiệt độ hơi có xu hướng giảm xuống do lượng hơi đi qua bộ quá nhiệt tăng lên trong khi lượng nhiệt cung cấp chưa kịp điều chỉnh tương ứng.

Nhiệt độ và lưu lượng khói lò cũng là những yếu tố nhiễu quan trọng. Sự dao động của quá trình đốt nhiên liệu, chất lượng nhiên liệu không đồng đều, hoặc thay đổi trong cơ chế cháy đều có thể làm thay đổi nhiệt độ khói lò, từ đó ảnh hưởng đến lượng nhiệt truyền cho hơi trong bộ quá nhiệt. Ngoài ra, áp suất hơi vào bộ quá nhiệt cũng có thể dao động do sự thay đổi trong quá trình sản xuất hơi tại lò hơi. Các yếu tố nhiễu này thường không thể đo lường trực tiếp và chính xác, đòi hỏi hệ thống điều khiển phải có khả năng bù trừ tác động của chúng thông qua cơ chế phản hồi mạnh.

\section{Thiết kế sơ đồ P\&ID}

\subsection{Nguyên tắc thiết kế}

Sơ đồ P\&ID (Piping and Instrumentation Diagram) là công cụ quan trọng để thể hiện đầy đủ cấu trúc của hệ thống điều khiển nhiệt độ hơi quá nhiệt. Sơ đồ này không chỉ mô tả các thiết bị công nghệ chính như bộ quá nhiệt, van điều khiển và đường ống hơi, mà còn thể hiện toàn bộ hệ thống đo lường, điều khiển và bảo vệ. Trong quá trình thiết kế, cần tuân thủ các tiêu chuẩn quốc tế về ký hiệu và quy ước để đảm bảo tính thống nhất và dễ hiểu.

Nguyên tắc cơ bản trong thiết kế sơ đồ P\&ID là phải thể hiện đầy đủ luồng công nghệ từ đầu vào đến đầu ra, bao gồm cả các nhánh phụ và hệ thống phụ trợ. Mỗi thiết bị đo lường cần được gắn nhãn rõ ràng với mã số định danh duy nhất, chỉ rõ loại đại lượng đo, vị trí lắp đặt và chức năng trong hệ thống điều khiển. Các van điều khiển cần được thể hiện với đầy đủ thông tin về kiểu van, cơ chế truyền động và chế độ hoạt động trong trường hợp mất nguồn cung cấp năng lượng.

% \begin{figure}[H]
% \centering
% % \includegraphics[width=\textwidth]{Hinhve/chuong1_so_do_pid_tong_quan.png}
% \caption{Sơ đồ P\&ID tổng quan của hệ thống điều khiển nhiệt độ hơi quá nhiệt}
% \label{fig:ch1_so_do_pid_tong_quan}
% \end{figure}

\subsection{Hệ thống đo lường}

Hệ thống đo lường trong sơ đồ P\&ID bao gồm các cảm biến nhiệt độ, áp suất và lưu lượng được bố trí tại các vị trí chiến lược. Cảm biến nhiệt độ chính được lắp đặt tại đầu ra của bộ quá nhiệt để đo trực tiếp nhiệt độ hơi quá nhiệt. Loại cảm biến thường được sử dụng là thermocouple hoặc RTD (Resistance Temperature Detector) do khả năng làm việc ổn định ở nhiệt độ cao và độ chính xác cao. Tín hiệu từ cảm biến nhiệt độ được chuyển đổi thành tín hiệu chuẩn 4-20 mA hoặc 0-10 V để truyền về bộ điều khiển PLC.

Cảm biến áp suất được lắp đặt tại các điểm trước và sau bộ quá nhiệt để giám sát chênh lệch áp suất, từ đó có thể phát hiện các hiện tượng bất thường như tắc nghẽn hoặc rò rỉ. Cảm biến lưu lượng, thường là loại orifice plate kết hợp với bộ chuyển đổi chênh áp, được sử dụng để đo lưu lượng hơi đi qua hệ thống. Ngoài ra, nhiệt độ khói lò cũng cần được giám sát thông qua các cảm biến nhiệt độ phù hợp, cung cấp thông tin về điều kiện cháy và lượng nhiệt có sẵn cho quá trình quá nhiệt.

% \begin{figure}[H]
% \centering
% % \includegraphics[width=0.85\textwidth]{Hinhve/chuong1_he_thong_do_luong.png}
% \caption{Bố trí hệ thống đo lường nhiệt độ, áp suất và lưu lượng}
% \label{fig:ch1_he_thong_do_luong}
% \end{figure}

\subsection{Thiết bị chấp hành và điều khiển}

Van điều khiển đóng vai trò là thiết bị chấp hành chính trong hệ thống, có nhiệm vụ điều chỉnh lưu lượng nước phun vào để giảm nhiệt độ hơi khi cần thiết. Van điều khiển thường là loại van điều tiết tuyến tính hoặc van bi có cơ chế điều khiển bằng khí nén hoặc động cơ điện. Đặc tính của van cần được lựa chọn phù hợp với dải điều chỉnh yêu cầu và đảm bảo độ tuyến tính tốt trong khoảng vận hành. Trong thiết kế, cần quan tâm đến vị trí an toàn của van khi mất tín hiệu điều khiển, thường được thiết kế ở trạng thái đóng để tránh hiện tượng nhiệt độ hơi giảm đột ngột.

Hệ thống điều khiển cascade được áp dụng để nâng cao chất lượng điều khiển nhiệt độ hơi quá nhiệt. Vòng điều khiển chính nhận tín hiệu từ cảm biến nhiệt độ hơi quá nhiệt và tính toán tín hiệu đặt cho vòng điều khiển phụ. Vòng điều khiển phụ có thể là vòng điều khiển lưu lượng nước phun hoặc vòng điều khiển nhiệt độ trung gian, có tốc độ đáp ứng nhanh hơn vòng chính. Cấu trúc cascade này giúp hệ thống phản ứng nhanh hơn với các nhiễu động tác động trực tiếp lên vòng phụ, đồng thời cải thiện độ ổn định tổng thể của hệ thống.

% \begin{figure}[H]
% \centering
% % \includegraphics[width=0.8\textwidth]{Hinhve/chuong1_cau_truc_dieu_khien_cascade.png}
% \caption{Cấu trúc điều khiển cascade cho hệ thống nhiệt độ hơi quá nhiệt}
% \label{fig:ch1_cau_truc_dieu_khien_cascade}
% \end{figure}

\subsection{Hệ thống bảo vệ}

Bên cạnh hệ thống điều khiển chính, sơ đồ P\&ID còn thể hiện các rơ-le bảo vệ và hệ thống an toàn. Các rơ-le nhiệt độ cao được lắp đặt để phát hiện và cảnh báo khi nhiệt độ hơi vượt quá giới hạn cho phép, đồng thời có thể kích hoạt các biện pháp bảo vệ như mở van xả nhiệt hoặc giảm tải tuabin. Rơ-le áp suất thấp và cao cũng được tích hợp để giám sát điều kiện áp suất trong hệ thống, ngăn ngừa các tình huống nguy hiểm có thể gây hư hỏng thiết bị.

Hệ thống liên động an toàn được thiết kế để đảm bảo các thiết bị chỉ vận hành khi các điều kiện tiên quyết được thỏa mãn. Ví dụ, hệ thống phun nước giảm nhiệt chỉ được phép hoạt động khi nhiệt độ hơi đạt ngưỡng nhất định và lưu lượng hơi đủ lớn để tránh hiện tượng ngưng tụ trong đường ống. Các van cách ly được bố trí tại các vị trí quan trọng để có thể tách riêng các phần của hệ thống khi cần bảo trì hoặc xử lý sự cố mà không ảnh hưởng đến toàn bộ nhà máy.

% \begin{figure}[H]
% \centering
% % \includegraphics[width=0.85\textwidth]{Hinhve/chuong1_he_thong_bao_ve.png}
% \caption{Sơ đồ hệ thống bảo vệ và liên động an toàn}
% \label{fig:ch1_he_thong_bao_ve}
% \end{figure}

\section{Xác định biến và thông số hệ thống}

\subsection{Phân loại các biến}

Trong hệ thống điều khiển nhiệt độ hơi quá nhiệt, việc xác định rõ ràng vai trò của từng biến là bước quan trọng để thiết lập chiến lược điều khiển hiệu quả. Biến điều khiển chính là nhiệt độ hơi quá nhiệt tại đầu ra của bộ quá nhiệt, đây là đại lượng cần được duy trì ổn định tại giá trị đặt trong mọi điều kiện vận hành. Độ chính xác yêu cầu đối với biến này thường rất cao, với sai số cho phép chỉ trong khoảng vài độ Celsius, do ảnh hưởng trực tiếp của nó đến hiệu suất và tuổi thọ của tuabin.

Biến điều khiển thứ cấp trong cấu trúc cascade có thể là lưu lượng nước phun giảm nhiệt hoặc nhiệt độ hơi tại vị trí trung gian trong bộ quá nhiệt. Lựa chọn biến điều khiển thứ cấp phụ thuộc vào đặc điểm cụ thể của thiết bị và yêu cầu về chất lượng điều khiển. Biến điều chỉnh của hệ thống là độ mở của van điều khiển nước phun, biến này được tác động trực tiếp bởi tín hiệu đầu ra từ bộ điều khiển và quyết định lượng nước phun vào dòng hơi để giảm nhiệt độ.

\subsection{Các yếu tố nhiễu}

Các biến nhiễu trong hệ thống được phân thành hai nhóm chính là nhiễu đo được và nhiễu không đo được. Nhiễu đo được bao gồm lưu lượng hơi đi qua bộ quá nhiệt, áp suất hơi vào và nhiệt độ hơi vào bộ quá nhiệt. Những nhiễu này có thể được đo lường thông qua các cảm biến và có khả năng áp dụng điều khiển tiến để bù trừ tác động của chúng. Lưu lượng hơi thay đổi theo tải của tuabin và được xem là nguồn nhiễu quan trọng nhất, có ảnh hưởng trực tiếp và đáng kể đến nhiệt độ hơi quá nhiệt.

Nhiễu không đo được bao gồm sự dao động của quá trình cháy nhiên liệu, thay đổi chất lượng nhiên liệu và các yếu tố môi trường như nhiệt độ không khí xung quanh. Những nhiễu này khó đo lường chính xác nhưng vẫn tác động đến quá trình quá nhiệt thông qua sự thay đổi của nhiệt độ và lưu lượng khói lò. Hệ thống điều khiển phản hồi phải đủ mạnh để bù trừ hiệu quả các nhiễu này mà không cần đo lường trực tiếp, dựa vào sai lệch giữa nhiệt độ thực tế và giá trị đặt.

% \begin{figure}[H]
% \centering
% % \includegraphics[width=0.8\textwidth]{Hinhve/chuong1_so_do_bien_dieu_khien_va_nhieu.png}
% \caption{Sơ đồ minh họa các biến điều khiển, điều chỉnh và nhiễu trong hệ thống}
% \label{fig:ch1_bien_dieu_khien_va_nhieu}
% \end{figure}

\subsection{Thông số kỹ thuật chính}

Các thông số quan trọng của hệ thống cần được xác định thông qua nghiên cứu tài liệu kỹ thuật của thiết bị và phân tích dữ liệu vận hành thực tế. Dải nhiệt độ hơi quá nhiệt làm việc thường nằm trong khoảng từ 500°C đến 560°C đối với các lò hơi áp suất trung bình. Áp suất hơi thường được duy trì ở mức 90-120 bar, tùy thuộc vào thiết kế của hệ thống. Lưu lượng hơi có thể thay đổi trong khoảng rộng từ 30% đến 100% tải định mức, đòi hỏi hệ thống điều khiển phải đảm bảo chất lượng tốt trong toàn bộ dải vận hành này.

Thời gian trễ của đối tượng điều khiển thường nằm trong khoảng từ 15-30 giây, trong khi hằng số thời gian có thể lên đến vài phút. Những thông số này sẽ được sử dụng làm cơ sở cho việc nhận dạng mô hình toán học của đối tượng điều khiển trong chương tiếp theo. Hệ số khuếch đại tĩnh của đối tượng, thể hiện mối quan hệ giữa sự thay đổi của lưu lượng nước phun và sự thay đổi nhiệt độ hơi ở trạng thái xác lập, cũng cần được xác định chính xác để thiết kế bộ điều khiển phù hợp.

\section{Kết luận chương}

Chương này đã trình bày hệ thống quá trình phân tích và thiết kế sơ đồ công nghệ cho hệ thống điều khiển nhiệt độ hơi quá nhiệt. Thông qua việc thu thập và phân tích các thông số công nghệ, đặc tính nhiệt động học của quá trình quá nhiệt đã được làm rõ, bao gồm tính chất trễ thời gian, quán tính nhiệt lớn và tính phi tuyến. Những đặc điểm này đặt ra thách thức đáng kể cho việc thiết kế hệ thống điều khiển, đồng thời cũng là cơ sở để lựa chọn phương pháp và cấu trúc điều khiển phù hợp.

Sơ đồ P\&ID chi tiết đã được thiết kế, thể hiện đầy đủ hệ thống đo lường, điều khiển, thiết bị chấp hành và hệ thống bảo vệ. Sơ đồ này không chỉ là công cụ để truyền đạt thông tin thiết kế mà còn là tài liệu quan trọng cho các giai đoạn lập trình, vận hành và bảo trì sau này. Việc xác định rõ ràng các biến điều khiển, biến điều chỉnh và các yếu tố nhiễu đã tạo nền tảng vững chắc cho việc phát triển mô hình toán học và tổng hợp bộ điều khiển trong chương tiếp theo.

Các thông số quan trọng của hệ thống đã được xác định, bao gồm dải nhiệt độ làm việc, áp suất, lưu lượng cũng như các đặc trưng động học như thời gian trễ và hằng số thời gian. Những thông tin này sẽ được sử dụng trong quá trình nhận dạng mô hình và thiết kế bộ điều khiển, đảm bảo hệ thống điều khiển cuối cùng có khả năng đáp ứng các yêu cầu kỹ thuật nghiêm ngặt của nhà máy nhiệt điện. Kết quả của chương này tạo điều kiện thuận lợi để chuyển sang giai đoạn thiết kế và tổng hợp bộ điều khiển, nơi các phương pháp điều khiển tiên tiến sẽ được áp dụng để đạt được hiệu suất điều khiển tối ưu.

\end{document}