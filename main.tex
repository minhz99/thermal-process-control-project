\documentclass[a4paper,13pt,3p,oneside]{report}
% \documentclass[a4paper,13pt,3p,twoside]{report}


% \usepackage{minted} % yêu cầu biên dịch với -shell-escape

\usepackage{booktabs}
\usepackage{mhchem}

\usepackage{scrextend}
\changefontsizes{13pt}
\usepackage[utf8]{vietnam}
\usepackage[top=2cm, bottom=2cm, left=3.5cm, right=2.5cm]{geometry}

\usepackage{tikz} % Thư viện vẽ hình
\usetikzlibrary{calc} % Add calc library for TikZ coordinates
\usepackage{graphicx} % Cho phép chèn hỉnh ảnh
\usepackage{fancybox} % Tạo khung box
\usepackage{indentfirst} % Thụt đầu dòng ở dòng đầu tiên trong đoạn
\usepackage{amsthm} % Cho phép thêm các môi trường định nghĩa
\usepackage{latexsym} % Các kí hiệu toán học
\usepackage{amsmath} % Hỗ trợ một số biểu thức toán học
\usepackage{amssymb} % Bổ sung thêm kí hiệu về toán học
\usepackage{amsbsy} % Hỗ trợ các kí hiệu in đậm
\usepackage{times} % Chọn font Time New Romans
\usepackage{array} % Tạo bảng array
\usepackage{enumitem} % Cho phép thay đổi kí hiệu của list
\usepackage{subfiles} % Chèn các file nhỏ, giúp chia các chapter ra nhiều file hơn
\usepackage{titlesec} % Giúp chỉnh sửa các tiêu đề, đề mục như chương, phần,..
\usepackage{titletoc}
\usepackage{chngcntr} % Dùng để thiết lập lại cách đánh số caption,..
\usepackage{pdflscape} % Đưa các bảng có kích thước đặt theo chiều ngang giấy
\usepackage{afterpage}
\usepackage{rotating}
\usepackage[ruled,vlined]{algorithm2e}  % Hỗ trợ viết các giải thuật
\usepackage{capt-of} % Cho phép sử dụng caption lớn đối với landscape page
\usepackage{multirow} % Merge cells
\usepackage{fancyhdr} % Cho phép tùy biến header và footer
% \usepackage[backend=biber,style=authoryear,maxnames=99,uniquename=false]{biblatex}

\usepackage{appendix}

\usepackage[font=small,labelfont=bf]{caption}
\usepackage[utf8]{inputenc} % Để hỗ trợ UTF-8

\usepackage{float}
\usepackage{subcaption}
\usepackage{xurl}

\usepackage[nonumberlist, nopostdot, nogroupskip, acronym]{glossaries}
\usepackage{glossary-superragged}
\setglossarystyle{superraggedheaderborder}
\usepackage{setspace}
\usepackage{parskip}

% package content table
\usepackage{tocbasic}

\usepackage{blindtext}
\usepackage[table,xcdraw]{xcolor}

\usepackage{listings} % Gói cho phép chèn code và định dạng code

% --- CẤU HÌNH HIỂN THỊ CODE ---
% ==== FONTS & UTF-8 (khuyên dùng XeLaTeX) ====
% \usepackage{fontspec}           % BẬT nếu dùng XeLaTeX/LuaLaTeX
% \setmonofont{JetBrains Mono}    % hoặc Inconsolata/Fira Code/… (tùy máy)

% Nếu dùng pdfLaTeX:
\usepackage[T5]{fontenc}
\usepackage[utf8]{inputenc}

% ==== LISTINGS & MÀU SẮC ====
\usepackage{listings}

% Bảng màu nhẹ, phù hợp in ấn
\definecolor{CodeBG}{rgb}{0.98,0.98,0.96}
\definecolor{CodeFrame}{rgb}{0.80,0.80,0.80}
\definecolor{CodeNum}{rgb}{0.55,0.55,0.55}
\definecolor{CodeCmt}{rgb}{0,0.45,0}
\definecolor{CodeStr}{rgb}{0.58,0,0.82}
\definecolor{CodeKw}{rgb}{0,0,0.6}
\definecolor{CodePunct}{rgb}{0.5,0.5,0.5}
\definecolor{CodeDelim}{RGB}{20,105,176}
\definecolor{CodeNumb}{RGB}{106,109,32}

% ==== STYLE CHUNG ====
\lstdefinestyle{tech}{
  backgroundcolor=\color{CodeBG},
%   basicstyle=\ttfamily\small,           % monospace, cỡ vừa phải cho đồ án
  basicstyle=\ttfamily\footnotesize,   % đổi từ \small sang \footnotesize
  numbers=left,
  numberstyle=\scriptsize\color{CodeNum},
  numbersep=8pt,
  frame=single,
  rulecolor=\color{CodeFrame},
  breaklines=true,
  tabsize=2,
  showstringspaces=false,
  keepspaces=true,
  captionpos=b,
  % Gợi ý màu
  keywordstyle=\color{CodeKw}\bfseries,
  commentstyle=\color{CodeCmt}\itshape,
  stringstyle=\color{CodeStr},
}

% ==== HỖ TRỢ DẤU & KÝ HIỆU KHOA HỌC (nếu pdfLaTeX) ====
\lstset{
  style=tech,
  extendedchars=true,
  inputencoding=utf8,
  literate=
   {đ}{{\dj}}1 {Đ}{{\DJ}}1
   {°}{{$^\circ$}}1 {Δ}{{$\Delta$}}1 {×}{{$\times$}}1 {≈}{{$\approx$}}1
   {ρ}{{$\rho$}}1
}

% ==== NGÔN NGỮ: JSON ====
\lstdefinelanguage{json}{
  basicstyle=\ttfamily\small,
  numbers=left, numberstyle=\scriptsize\color{CodeNum}, numbersep=8pt,
  showstringspaces=false,
  breaklines=true,
  frame=single, rulecolor=\color{CodeFrame},
  morestring=[b]",
  commentstyle=\color{CodeCmt}\itshape,
  stringstyle=\color{CodeStr},
  morecomment=[l]{//},
  literate=
   *{0}{{{\color{CodeNumb}0}}}{1}
    {1}{{{\color{CodeNumb}1}}}{1}
    {2}{{{\color{CodeNumb}2}}}{1}
    {3}{{{\color{CodeNumb}3}}}{1}
    {4}{{{\color{CodeNumb}4}}}{1}
    {5}{{{\color{CodeNumb}5}}}{1}
    {6}{{{\color{CodeNumb}6}}}{1}
    {7}{{{\color{CodeNumb}7}}}{1}
    {8}{{{\color{CodeNumb}8}}}{1}
    {9}{{{\color{CodeNumb}9}}}{1}
    {:}{{{\color{CodePunct}{:}}}}{1}
    {,}{{{\color{CodePunct}{,}}}}{1}
    {\{}{{{\color{CodeDelim}{\{}}}}{1}
    {\}}{{{\color{CodeDelim}{\}}}}}{1}
    {[}{{{\color{CodeDelim}{[}}}}{1}
    {]}{{{\color{CodeDelim}{]}}}}{1},
}

% ==== NGÔN NGỮ: InfluxQL (InfluxDB 1.x) ====
\lstdefinelanguage{InfluxQL}{
  morekeywords={
    SELECT,FROM,WHERE,AND,OR,GROUP,BY,ORDER,LIMIT,SLIMIT,SOFFSET,OFFSET,INTO,MEASUREMENT,SHOW,MEASUREMENTS,RETENTION,POLICIES,CONTINUOUS,QUERIES,ON,CREATE,DELETE,DROP,FIELD,KEYS,TIME,ASC,DESC,FILL
  },
  morekeywords=[2]{mean,max,min,sum,count,first,last,median,percentile,derivative,non_negative_derivative,difference,moving_average,stddev,spread},
  sensitive=false,
  keywordstyle=\color{CodeKw}\bfseries,
  keywordstyle=[2]\color{CodeKw},
  comment=[l]{--},
  morecomment=[l]{//},
  morestring=[b]',
  morestring=[b]",
}

% ==== NGÔN NGỮ: Flux (InfluxDB 2.x) ====
\lstdefinelanguage{Flux}{
  morekeywords={
    import,option,from,range,filter,group,aggregateWindow,mean,max,min,sum,count,keep,drop,map,pivot,sort,yield,join,union,duplicate,rename,window,fill
  },
  sensitive=true,
  keywordstyle=\color{CodeKw}\bfseries,
  comment=[l]{//},
  morestring=[b]",
}

% ==== TIỆN DỤNG: môi trường rút gọn theo ngôn ngữ ====
\lstnewenvironment{pycode}[1][]{\lstset{language=Python,#1}}{}
\lstnewenvironment{cppcode}[1][]{\lstset{language=C++,#1}}{}
\lstnewenvironment{jsoncode}[1][]{\lstset{language=json,#1}}{}
\lstnewenvironment{influxql}[1][]{\lstset{language=InfluxQL,#1}}{}
\lstnewenvironment{fluxcode}[1][]{\lstset{language=Flux,#1}}{}

 % ===================================================

\renewcommand{\bibname}{Danh_sach_tai_lieu_tham_khao} 
% \usepackage[backend=bibtex,style=ieee]{biblatex}  %backend=biber is 'better'
\usepackage[natbib,backend=biber,style=ieee]{biblatex} % Giúp chèn tài liệu tham khảo
% \usepackage[backend=biber,style=authoryear,maxnames=99,uniquename=false]{biblatex}
% \DefineBibliographyStrings{english}{%
%     and = {và},
%     volume = {tập},
% }

\renewcommand\appendixname{PHỤ LỤC}
\renewcommand\appendixpagename{PHỤ LỤC}
\renewcommand\appendixtocname{PHỤ LỤC}

\addbibresource{Danh_sach_tai_lieu_tham_khao.bib} % chèn file chứa danh mục tài liệu tham khảo vào

% !TeX root = ../main.tex

%\makeglossaries
\makenoidxglossaries

% Tạo glossary mới cho ký hiệu
% \newglossary[symg]{symbols}{syi}{syo}{DANH SÁCH KÝ HIỆU}

% Danh mục thuật ngữ và từ viết tắt
\newglossaryentry{BGNHA}{
    type=\acronymtype,
    name={BGNHA},
    description={Bình gia nhiệt hạ áp},
    first={Bình gia nhiệt hạ áp (BGNHA)},
}

\newglossaryentry{BGNCA}{
    type=\acronymtype,
    name={BGNCA},
    description={Bình gia nhiệt cao áp},
    first={Bình gia nhiệt cao áp (BGNCA)},
}

\newglossaryentry{BKK}{
    type=\acronymtype,
    name={BKK},
    description={Bình khử khí},
    first={Bình khử khí (BKK)},
}

\newglossaryentry{QNTG}{
    type=\acronymtype,
    name={QNTG},
    description={Quá nhiệt trung gian},
    first={Quá nhiệt trung gian (QNTG)},
}

\newglossaryentry{FGD}{
    type=\acronymtype,
    name={FGD},
    description={Hệ thống xử lý khí thải khử lưu huỳnh - Flue Gas Desulfurization},
    first={Flue Gas Desulfurization - FGD},
}
% \newglossaryentry{delta}{
%     type=symbols,
%     name={$\Delta$},
%     description={Độ chênh lệch}
% }

% \newglossaryentry{t}{
%     type=symbols,
%     name={$t$},
%     description={Nhiệt độ}
% }

% \newglossaryentry{eta}{
%     type=symbols,
%     name={$\eta$},
%     description={Hiệu suất}
% }

% ===================================================


\fancypagestyle{plain}{%
\fancyhf{} % clear all header and footer fields
\fancyfoot[RO,RE]{\thepage} %RO=right odd, RE=right even
\renewcommand{\headrulewidth}{0pt}
\renewcommand{\footrulewidth}{0pt}}

\setlength{\headheight}{10pt}

\def \TITLE{ĐỒ ÁN TỐT NGHIỆP}
\def \AUTHOR{VŨ ĐỨC MINH}

% ===================================================
\titleformat 
    {\chapter} % command
    [hang] % shape
    {\centering\bfseries} % format
    {CHƯƠNG \thechapter.\ } % label
    {0pt} %sep
    {} % before
    [] % after
\titlespacing*{\chapter}{0pt}{-20pt}{20pt}

\titleformat
    {\section} % command
    [hang] % shape
    {\bfseries} % format
    {\thechapter.\arabic{section}\ \ \ \ } % label
    {0pt} %sep
    {} % before
    [] % after
\titlespacing{\section}{0pt}{\parskip}{0.5\parskip}

\titleformat
    {\subsection} % command
    [hang] % shape
    {\bfseries} % format
    {\thechapter.\arabic{section}.\arabic{subsection}\ \ \ \ } % label
    {0pt} %sep
    {} % before
    [] % after
\titlespacing{\subsection}{30pt}{\parskip}{0.5\parskip}

\renewcommand\thesubsubsection{\alph{subsubsection}}
\titleformat
    {\subsubsection} % command
    [hang] % shape
    {\bfseries} % format
    {\alph{subsubsection}, \ } % label
    {0pt} %sep
    {} % before
    [] % after
\titlespacing{\subsubsection}{50pt}{\parskip}{0.5\parskip}

\newcommand{\underwrite}[3][]{% \underwrite[<thickness>]{<numerator>}{<denominator>}
  \genfrac{}{}{#1}{}{\textstyle #2}{\scriptstyle #3}
}
% \newcommand{\titlesize}{\fontsize{18pt}{23pt}\selectfont}
% \newcommand{\subtitlesize}{\fontsize{16pt}{21pt}\selectfont}
% \titleclass{\part}{top}
% \titleformat{\part}[display]
%   {\normalfont\huge\bfseries}{\centering}{20pt}{\Huge\centering}
% \titlespacing{\part}{0pt}{em}{1em}
% \titlespacing{\section}{0pt}{\parskip}{0.5\parskip}
% \titlespacing{\subsection}{0pt}{\parskip}{0.5\parskip}
% \titlespacing{\subsubsection}{0pt}{\parskip}{0.5\parskip}

% ===================================================
\usepackage{hyperref}
\hypersetup{pdfborder = {0 0 0}} %
\hypersetup{pdftitle={\TITLE},
	pdfauthor={\AUTHOR}}
	
\usepackage[all]{hypcap} % Cho phép tham chiếu chính xác đến hình ảnh và bảng biểu

\graphicspath{{figures/}{../figures/}} % Thư mục chứa các hình ảnh

\counterwithin{figure}{chapter} % Đánh số hình ảnh kèm theo chapter. Ví dụ: Hình 1.1, 1.2,..

\title{\bf \TITLE}
\author{\AUTHOR}

\setcounter{secnumdepth}{3} % Cho phép subsubsection trong report
% \setcounter{tocdepth}{3} % Chèn subsubsection vào bảng mục lục

\theoremstyle{definition}
\newtheorem{example}{Ví dụ}[chapter] % Định nghĩa môi trường ví dụ

\onehalfspacing
%Khoảng cách xuống dòng
\setlength{\parskip}{6pt}
%Lùi đầu dòng
\setlength{\parindent}{15pt}



% =========================== BODY ===============
\begin{document}
% \newgeometry{top=2cm, bottom=2cm, left=2cm, right=2cm}
\subfile{Bia} % Phần bìa
% \restoregeometry

% ===================================================
\pagenumbering{roman}

% \newpage
% \pagenumbering{gobble}
% \subfile{Chuong/0_0_Loi_cam_doan.tex}

\newpage
\pagenumbering{gobble}
\subfile{Chuong/0_1_Loi_cam_on.tex}

\newpage
\pagenumbering{gobble}
\subfile{Chuong/0_2_Tom_tat_noi_dung.tex}

\newpage
\pagenumbering{gobble}
\subfile{Chuong/0_3_Loi_noi_dau.tex}

% ===================================================
% \pagestyle{empty} % Header và footer rỗng
\newpage
\pagenumbering{roman} % Xóa page numbering ở cuối trang
\pagestyle{plain}
\renewcommand*\contentsname{MỤC LỤC}

\titlecontents{chapter}
    [0.0cm]             % left margin
    {\bfseries\vspace{0.3cm}}                  % above code
    {{\bfseries{\scshape}
    CHƯƠNG \thecontentslabel.\ }}
    % numbered format
    {}         % unnumbered format
    {\titlerule*[0.3pc]{.}\contentspage}         % filler-page-format, e.g dots

    
\titlecontents{section}
    [0.0cm]             % left margin
    {\vspace{0.3cm}}                  % above code
    {\thecontentslabel \ } % numbered format
    {}         % unnumbered format
    {\titlerule*[0.3pc]{.}\contentspage}         % filler-page-format, e.g dots
    
\titlecontents{subsection}
    [1.0cm]             % left margin
    {\vspace{0.3cm}}                  % above code
    {\thecontentslabel \ } % numbered format
    {}         % unnumbered format
    {\titlerule*[0.3pc]{.}\contentspage}         % filler-page-format, e.g dots

 % Tạo mục lục tự động
\addtocontents{toc}{\protect\thispagestyle{plain}}
\tableofcontents 
\thispagestyle{plain}
\cleardoublepage

% \pagenumbering{roman}
%Tạo danh mục hình vẽ.
\pagestyle{plain}
\renewcommand{\listfigurename}{DANH MỤC HÌNH VẼ}
{\let\oldnumberline\numberline
\renewcommand{\numberline}{Hình~\oldnumberline}
\listoffigures} 
\thispagestyle{plain}
% \phantomsection\addcontentsline{toc}{section}{\numberline {} DANH MỤC HÌNH VẼ}
% \newpage


 %Tạo danh mục bảng biểu.
\pagestyle{plain}
\renewcommand{\listtablename}{DANH MỤC BẢNG BIỂU}
{\let\oldnumberline\numberline
\renewcommand{\numberline}{Bảng~\oldnumberline}
\listoftables}
\thispagestyle{plain}
% \phantomsection\addcontentsline{toc}{section}{\numberline {} DANH MỤC BẢNG BIỂU}

% \glsaddall
\renewcommand*{\acronymname}{}
\renewcommand*{\descriptionname}{Ý nghĩa}

\renewcommand*{\entryname}{Thuật ngữ/từ viết tắt}

\setglossarystyle{long} % dùng longtable 2 cột, tự xuống dòng (long là style tích hợp của glossaries)
\renewcommand{\arraystretch}{1.5} % (tuỳ chọn) giãn dòng một chút cho dễ đọc

\printnoidxglossary[type=acronym,
                    title={DANH MỤC TỪ VIẾT TẮT},
                    toctitle={Danh sách thuật ngữ/viết tắt}]

% \renewcommand*{\entryname}{Ký hiệu}
% \printnoidxglossary[type=symbols,
%                     title={DANH MỤC KÝ HIỆU},
%                     toctitle={Danh sách ký hiệu}]

% ===================================================


\newpage
\pagenumbering{arabic}

\pagestyle{fancy}
\fancyhf{}
\renewcommand{\headrulewidth}{0.4pt}
\setlength{\headheight}{14pt}
\makeatletter
\renewcommand{\chaptermark}[1]{\markboth{CHƯƠNG \thechapter.\ #1}{}}
\makeatother
\fancyhead[LO]{\nouppercase{\leftmark}}
\fancyhead[RE]{\nouppercase{\leftmark}}
\fancyfoot[RO]{\thepage}
\renewcommand{\sectionmark}[1]{}

\chapter{CƠ SỞ CÔNG NGHỆ VÀ XÂY DỰNG MÔ HÌNH ĐIỀU KHIỂN PID}
\label{chuong1}
\subfile{Chuong/1_Chuong_1.tex} % Phần mở đầu

\newpage
% \pagestyle{fancy} % Áp dụng header và footer
\chapter{NHẬN DẠNG ĐỐI TƯỢNG, THIẾT KẾ VÀ CHỈNH ĐỊNH BỘ ĐIỀU KHIỂN}
\label{chuong2}
\subfile{Chuong/2_Chuong_2.tex}

\newpage
% \pagestyle{fancy} % Áp dụng header và footer
\chapter{TRIỂN KHAI VÀ THỰC THI HỆ THỐNG ĐIỀU KHIỂN TRÊN PLC S7-1200}
\label{chuong3}
\subfile{Chuong/3_Chuong_3.tex}

\newpage
% \pagestyle{fancy} % Áp dụng header và footer
\chapter{KẾT LUẬN VÀ HƯỚNG PHÁT TRIỂN}
\label{chuong4}
\subfile{Chuong/4_Chuong_4.tex}

% \newpage
% % \pagestyle{fancy} % Áp dụng header và footer
% \chapter{KẾT LUẬN VÀ HƯỚNG PHÁT TRIỂN}
% \label{chuong5}
% \subfile{Chuong/5_Chuong_5.tex}

% ===================================================

\newpage
\renewcommand\bibname{TÀI LIỆU THAM KHẢO}
\printbibliography
\phantomsection\addcontentsline{toc}{chapter}{TÀI LIỆU THAM KHẢO}

% \appendixpage
% \appendices
% \addappheadtotoc

% % \chapter*{PHỤ LỤC} %Kết luận và hướng phát triển}

% %\mainmatter
% \titleformat{\chapter}[hang]{\centering\bfseries}{ \thechapter.\ }{0pt}{}[]
% \titlespacing*{\chapter}{0pt}{-20pt}{20pt}

% \titlecontents{chapter}
%     [0.0cm]             % left margin
%     {\bfseries\vspace{0.3cm}}                  % above code
%     {{\bfseries{\scshape} \thecontentslabel.\ }} % numbered format
%     {}         % unnumbered format
%     {\titlerule*[0.3pc]{.}\contentspage}         % filler-page-format, e.g dots
% \chapter{MÃ NGUỒN VÀ TÀI LIỆU KỸ THUẬT}
% \subfile{Chuong/Phu_luc_A}
% \newpage
% \chapter{THÔNG SỐ THIẾT KẾ MÔ HÌNH}
% \subfile{Chuong/Phu_luc_B}
\end{document}
